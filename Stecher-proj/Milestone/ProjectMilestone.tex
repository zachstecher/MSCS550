\documentclass[12pt]{article}

\topmargin 0.0cm
\oddsidemargin 0.2cm
\textwidth 16cm 
\textheight 21cm
\footskip 1.0cm

\title{Project Milestone}
\author{Zach Stecher}
\date{Due: 11/15/16}

\begin{document}

\maketitle

\section*{\centering{Abstract}}

I discuss the problem of solving image-based classification using readily available machine learning techniques. Specifically, I discuss the issue of accurately classifying 99 species of plants using binary leaf images with multiple features. For this project I compare the results achieved by different available algorithms to find the most effective solution, and discuss real world applications and ease of practical use. The three initial algorithms I would like to analyze are the Multi-Layered Perceptron(MLP), TensorFlow Neural Network, and Random Forest.

\section{Introduction}

The focus of this project is the classificaton of provided leaf images and corresponding data into one of a possible 99 species of plants (https://www.kaggle.com/c/leaf-classification). While this is generally regarded as a simple problem to solve using existing standard techniques, there is always opportunity to expand our understanding of existing problems and techniques by applying and comparing them. For this project I aim to compare a number of different classifiers for this problem to derive which can achieve the most accurate results, and the speed of calculation for each one.

\subsection{Practical Application}

Automated plant recognition as a subset of general image based classification has a multitude of real world applications. Beyond just the academic implications, an accurate classifier open to the public could aid in applications like medicinal research, species tracking, agricultural research, medicinal practice in low-income countries where manufactured pharmaceutical solutions are not available, and much more.

\subsection{}


\end{document}
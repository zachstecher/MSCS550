\documentclass[11pt]{article}
\usepackage[letterpaper, margin=1in]{geometry}
\usepackage{listings}
\usepackage{graphicx}
\usepackage{amsmath}
\usepackage{amsfonts}

\title{Homework 01}
\author{Zach Stecher}
\date{Due: 9/20/16}

\begin{document}
\lstset{language=Python}

\maketitle

\section*{Problem 1.2}
\subsection*{1.2a)}

We know that the regions represented h(x) = +1 and h(x) = -1 can be seperated by a line because h(x) = 0 seperates the two reasons, so the line that seperates the two regions is h(x) = 0.
\\

h(x) = 0 can be represented as:

\begin{equation}
	x_0w_0 + x_1w_1 + x_2w_2 = 0
\end{equation}
\\

By moving some of the equation to the other side, we end up with:

\begin{equation}
	x_2 = \frac{-w_0 - x_1w_1}{w_2}
\end{equation}

From there, if we move $-w_0$ to the end and split this side of the equation up for clarity, we get:

\begin{equation}
	x_2 = \frac{-x_1w_1}{w_2} - \frac{w_0}{w_2}
\end{equation}

So using $y = mx + b$ where $m$ is the slope and $b$ is the intersect, in terms of $w$ and $x$ the slope is $\frac{-x_1}{w_2}$ and the intersect is $\frac{-w_0}{w_2}$.
\\

\subsection*{1.2b Draw figures for cases w = [1, 2, 3] and w = -[1, 2, 3]}
\begin{figure}
	\includegraphics{12bpos.png}
	\caption{With positive weight values.}
\end{figure}

\begin{figure}
	\includegraphics{12bneg.png}
	\caption{With negative weight values}.
\end{figure}
\newpage

\section*{Problem 1.4}

For all sections of problem 1.4, we used a provided base version of the Perceptron learning algorithm with small modifications.

\subsection*{1.4a}

For this section, the Perceptron was called on a deta set size 20 by the following code:

\begin{lstlisting}[frame=single]
p = Perceptron(20)
p.plot()
\end{lstlisting}

\begin{figure}[!htb]
	\includegraphics{14a.png}
	\caption{The final hypothesis $h(g)$.}
\end{figure}
\newpage

\subsection*{1.4b}

For this section, we ran the Perceptron algorithm again on a data set with a size of 20, taking note of the number of iterations necessary to reach $h(g)$. For this instance, we ended up with 17 iterations to reach $h(g)$.
The code was modified to save all iterations as .png files:

\begin{lstlisting}[frame=single]
p = Perceptron(20)
p.pla(save=True)
p.plot()
\end{lstlisting}

\begin{figure}[!htb]
	\includegraphics{14b.png}
	\caption{The final hypothesis $h(g)$ after 17 iterations.}
\end{figure}
\newpage

\subsection*{1.4c Run the Perceptron again with a data set size of 20 and compare results with 1.4b}

This time the Perceptron algorithm only required 14 iterations to find $h(g)$. The code was not modified at all for this section.

\begin{figure}[!htb]
	\includegraphics{14c.png}
	\caption{The final hypothesis h(g) after 14 iterations.}
\end{figure}
\newpage

\subsection*{1.4d Run the Perceptron again with a data set size of 100 and compare results.}

This time we ended up with 31 iterations and a much longer runtime to reach $h(g)$. It's looking like the runtimes get exponentially longer the more pieces of data introduced, as each entry may need to be re-checked for every iteration.

\begin{lstlisting}[frame=single]
p = Perceptron(100)
p.pla(save=True)
p.plot()
\end{lstlisting}

\begin{figure}[!htb]
	\includegraphics{14d.png}
	\caption{The final hypothesis $h(g)$ after 31 iterations.}
\end{figure}
\newpage

\subsection*{1.4e Run the Perceptron again with data set size of 1000. Compare results}

This section actually managed to freeze my computer after it had arrived at $h(g)$. It only required 49 iterations, but the runtime was significantly longer than any of the other experiments.

\begin{lstlisting}[frame=single]
p = Perceptron(1000)
p.pla(save=True)
p.plot()
\end{lstlisting}

\begin{figure}[!htb]
	\includegraphics{14e.png}
	\caption{The final hypothesis $h(g)$ after 49 iterations.}
\end{figure}

\end{document}
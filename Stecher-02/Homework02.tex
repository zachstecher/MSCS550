\documentclass[11pt]{article}
\usepackage{amsmath}
\usepackage{amsfonts}
\usepackage{listings}

\title{Homework 02}
\author{Zach Stecher}
\date{Due: 9/27/16}

\begin{document}

\maketitle

\section*{Problem 2.1}
\subsection*{2.1a)  For $M$ = 1, how many examples do we need to make \\ $\varepsilon$ $\leq$ 0.05?}

\begin{equation*}
\varepsilon(M, N, \delta) = \sqrt{\frac{1}{2N}ln\frac{2M}{\delta}}
\end{equation*}

By subbing the provided numbers into the above forumla, we can begin solving for $N$. First we'll plug in the numbers:

\begin{equation*}
\sqrt{\frac{1}{2N}ln\frac{2(1)}{0.03}} \leq 0.05
\end{equation*}

Then we math out the right side of the part of the forumla inside the square root:

\begin{equation*}
\sqrt{\frac{1}{2N} x 4.1999} \leq 0.05
\end{equation*}

Then we start to isolate the N. First we square both sides of the equation to remove the square root, resulting in:

\begin{equation*}
\frac{1}{2N} x 4.1999 \leq 0.0025
\end{equation*}

Then we move $\frac{1}{2N}$ to the other side and multiply by 2 to remove the fraction. Eventually we're left with:

\begin{equation*}
\frac{4.199}{0.005} \leq N
\end{equation*}

So we get $N$ $\geq$ $840$. We need at minimum 840 examples to achieve the desired error tolerance.

\subsection*{2.1b)  For $M$ = 100, how many examples do we need to make \\ $\varepsilon$ $\leq$ 0.05?}

Using the same equation and process, but subbing 100 in for $M$, we get the answer $N$ $\geq$ $1761$.

\subsection*{2.1c)  For $M$ = 100, how many examples do we need to make \\ $\varepsilon$ $\leq$ 0.05?}

Using the same equation and process, but subbing 1000 in for $M$, we get the answer $N$ $\geq$ $2683$.

\section*{Problem 2.11  Suppose $m_H$$(N)$ $=$ $N$ + $1$, so $d_{vc}$ $=$ $1$. With 100 training examples, give a bound for $E_{out}$ with confidence 90 percent. Repeat for $n$ $=$ $10,000$.}

\begin{equation*}
E_{out}(g) \leq E_{in}(g) + \sqrt{\frac{8}{2N} ln \frac{4m_h(N)}{\delta}}
\end{equation*}

By plugging the provided numbers into the equation, we get this (simplified):

\begin{equation*}
\sqrt{\frac{8}{100} ln \frac{804}{0.1}}
\end{equation*}

By mathing this out, we get the bound (rounded up to the nearest hundreth):

\begin{equation*}
E_{out}(g) \leq E_{in}(g) + 0.85.
\end{equation*}

Repeating for $N$ $=$ $10,000$ we get:

\begin{equation*}
E_{out}(g) \leq E_{in}(g) + 0.14
\end{equation*}

\section*{Problem 2.12  For an $H$ with $d_vc$ = 10, what sample size do you need to have a 95 percent confidence that your  generalization error is at most 0.05?}
\begin{equation*}
N \geq \frac{8}{\varepsilon^2} ln (\frac{4((2N)^{d_{vc}} + 1)}{\delta})
\end{equation*}

Because N exists on both sides of the above equation, we can use it to iteratively solve this problem. First, we plug in the correct numbers:

\begin{equation*}
N \geq \frac{8}{0.05^2} ln (\frac{4((2N)^{10} + 1)}{0.05})
\end{equation*}

Then we can begin iterating to find the answer. Let's start with N = 1,000.

\begin{equation*}
N \geq \frac{8}{0.05^2} ln (\frac{4((2 * 1000)^{10} + 1)}{0.05})
\end{equation*}

If we math this out, we get $N$ $\geq$ $257251.363936$. From here, we plug $257,251.363936$ in for N in the equation we just used and do it again, arriving at $361,170.64168$. Eventually the results start evening out just below $453,000$, so we can say that $N$ $\approx$ $453,000$.

\end{document}